% jqmath.tex  -  defines useful macros for math typesetting
%
% Required packages: amsmath, amssymb, mathtools
%
% Author: 			Jay Y. Qi
% Last modified:	March 11, 2015

% Parentheses and brackets
\DeclarePairedDelimiter\paren{(}{)}
\DeclarePairedDelimiter\sbrack{[}{]}
\DeclarePairedDelimiter\cbrack{\{}{\}}
\DeclarePairedDelimiter\abrack{\langle}{\rangle}

% Number sets
\newcommand{\realset}{\mathbb{R}}
\newcommand{\complexset}{\mathbb{C}}
\newcommand{\integerset}{\mathbb{Z}}
\newcommand{\rationalset}{\mathbb{Q}}

% Real and imaginary operators (redefine with roman letters)
\let\Re\relax
\let\Im\relax
\DeclareMathOperator{\Re}{Re}
\DeclareMathOperator{\Im}{Im}

% Derivatives
\newcommand{\deriv}[2]{\frac{d#1}{d#2}}
\newcommand{\pderiv}[2]{\frac{\partial#1}{\partial#2}}

% Vertical bar for evaluated at
\newcommand{\evalat}[2]{\left. #1 \right\vert_{#2}}

% Norms and inner products (requires math tools)
\DeclarePairedDelimiter\abs{\lvert}{\rvert}
\DeclarePairedDelimiter\norm{\lVert}{\rVert}
\DeclarePairedDelimiterX\inner[2]{\langle}{\rangle}{#1,#2}

% Symbols
\let \by \times
\let \del \nabla
\newcommand{\given}{\middle\vert}